%
% File acl2017.tex
%
%% Based on the style files for ACL-2015, with some improvements
%%  taken from the NAACL-2016 style
%% Based on the style files for ACL-2014, which were, in turn,
%% based on ACL-2013, ACL-2012, ACL-2011, ACL-2010, ACL-IJCNLP-2009,
%% EACL-2009, IJCNLP-2008...
%% Based on the style files for EACL 2006 by 
%%e.agirre@ehu.es or Sergi.Balari@uab.es
%% and that of ACL 08 by Joakim Nivre and Noah Smith

\documentclass[11pt,a4paper]{article}
\usepackage[hyperref]{acl2017}
\usepackage{times}
\usepackage{latexsym}

\usepackage{url}

%\aclfinalcopy % Uncomment this line for the final submission
%\def\aclpaperid{***} %  Enter the acl Paper ID here

%\setlength\titlebox{5cm}
% You can expand the titlebox if you need extra space
% to show all the authors. Please do not make the titlebox
% smaller than 5cm (the original size); we will check this
% in the camera-ready version and ask you to change it back.

\newcommand\BibTeX{B{\sc ib}\TeX}

\title{CMPUT 497 Project Draft Report: \\ RAKE - Key Word Extraction Replication}

\author{Shouyang Zhou \\
  University of Alberta \\
  Edmonton, Alberta, Canada \\
  {\tt shoyang@ualberta.ca} \\\And
  Sharon Hains \\
  University of Alberta \\
  Edmonton, Alberta, Canada \\
  {\tt hains@ualberta.ca} \\\And
  Sharif Bakouny \\
  University of Alberta \\
  Edmonton, Alberta, Canada \\
  {\tt albakoun@ualberta.ca} \\}


\date{}

\begin{document}
\maketitle

\section{Introduction}

We aim to replicate the main evaluation from the article "Automatic Keyword Extraction from Individual Documents" by Rose et all. This paper devises an unsupervised method for keyword extraction titled "RAKE" and compares it to a previous well performing unsupervised method called "TextRank". Details to follow in the evaluation section.

Keyword extraction is the automated process of extracting important words and phrases from a document. The importance of keyword extraction is in application, in information retrieval, feature engineering, and augmenting human classification tasks. At its most basic level, it helps humans process unstructured data in a more efficient and digestible manner. The importance of this replication study is to verify the results of \citet{1}, so we can be more confident in using RAKE as keyword extraction method. Confirming the results of RAKE testing is relevant as keyword extraction is a widely used application as discussed above. 

\subsection{Input-Output}

The overarching output will be a table in the form of Table 1.2 in \citet{1}, please see the appendix for details. This will be a table comparing the performance of RAKE and TextRank variants (parameters) listing the metrics: extracted keywords (total, mean), correct keywords (total, mean), precision, recall, f-measure summarizing the replication-evaluation.

The unit output per trial will be a simple experiment comparing the keywords extracted by RAKE and TextRank variants by correctly extracted keywords, correct keywords, precision, recall, and f-measure over an input text and input sequence of "truth/reference" keywords. 

\section{Related Work}

The original paper “Automatic Keyword Extraction from Individual Documents” by \citet{1} is the primary work of interest. As mentioned, they evaluate their method "RAKE" and compare it with "TextRank" over two datasets. RAKE requires a set of phrase and content-word delimiters called a stoplist, a list of stopwords. Stop words are punctuation, numbers, conjuctions, and user specified terms which are used to delimit candidate keyword/phrases. 

A brief summary of the RAKE algorithm:
\begin{enumerate}
\item Split the text into an array of words using the word delimiters.
\item Split the array into sequences contiguous words using stop words and phrase delimiters.
\item Candidate keywords are words in a sequence that are assigned the same position in the text. 
\item Assign scores to each keyword candidate using ratio of degree to word-frequency. 
\item Keywords that contain stop words:
\begin{enumerate}
\item A pair of candidate keywords must be adjoined at least twice in the text in the same order.
\item Create a new keyword which contains the pair of keywords with interior stopwords between them.
\item The new keyword’s score is the sum of the scores of its keywords components.
\end{enumerate}
\item The keywords of the text are the top T keywords from the keyword candidates list.
\end{enumerate}

\citet{1} also develops methods for stoplist generation. These stoplist generation methods leverage supervised datasets to generate dataset specific (thus domain specific) stoplists. 

RAKE was compared to TextRank and seminal supervised learning methods (Hulth 2003) over a dataset of human keyword annotated scientific paper abstracts originating from \citet{hulth-2003-improved}. The performance of RAKE depends on the stoplist used. RAKE was found to outperform all previously used keyword extraction algorithms in precision, efficiency and simplicity when using a domain specific stoplist. Using a generic stoplist, RAKE was found to be no worse performing than TextRank.

\citet{4} discusses TextRank, a graph-based ranking algorithm for keyword extraction, where the importance of a vertex (phrases) is decided by considering global information value of the phrase recursively computed from the entire graph (text). Implementing this algorithm goes as follows:

\begin{enumerate}
\item Identify text units and add them as vertices to the graph.
\item Edges of the graph are relations between text units.
\item Iterate the algorithm until convergence below a given threshold. 
\item Rank vertices based on their final scores (values).
\end{enumerate}   

The authors of TextRank evaluate their method to \citet{hulth-2003-improved} using the same dataset in RAKE, again, originating from \citet{hulth-2003-improved}.

We will not be recreating the results of \citet{hulth-2003-improved}, however as we are including their results in our evaluation for reference. \citet{1} describe their method where "Hulth (2003) compares the effectiveness of three term selection approaches: noun-phrase (NP) chunks, n-grams, and POS tags, with four discriminative features of these terms as inputs for automatic keyword extraction using a supervised machine-learning algorithm." \citet{1} noted difficulty in finding training materials used by \citet{hulth-2003-improved}.    

\section{Methodology}

Since our inital proposal, we have found implmentations of RAKE and TextRank in python (\citet{2}, \citet{3}). Since these are already availible to us, we will replicate \citet{1} Rose et all's evaluation of the two using third party libraries. 

In essence, we will implment an evaluation script that feeds a dataset into these third party libraries to extract then aggregate the resultant metrics. This task will involve data collection, understanding the interface to the RAKE and textrank implementations, preprocessing datasets to be fed into the two methods, and extracting and aggregating the results.

\section{Evaluation}

As our project is based on replicating the results of an article, we will evaluate the aggregate measures recorded as per the inital study and conduct an error analysis from samples from the replicating evaluation. 

Again, the measures reported in the original study by method were: number of extracted keywords, correct number of extracted keywords, precision, recall, and f-measure (f-score).  We will compare our recorded measures to that of the original study to what extent are Rose et all's results reproducible. We will consider Rose et all's results reproduceable if the ordinal performance between RAKE and TextRank variants can be verified. 

For this reproduction we require three datasets:

\begin{enumerate}
\item \citet{hulth-2003-improved}'s dataset of human keyword annotated scientific paper abstracts.
\item Fox's Stoplist, a generic stopword-list. 
\item Keyword Adjecency stoplist, a stoplist generated by the authors of RAKE.
\end{enumerate}   

Thus far, we have found the dataset used by \citet{hulth-2003-improved} and Fox's Stoplist.

\section{Remaining Work}

We have collected the referenced datasets and mapped out pseudo code, as mentioned above in our Implementation section, to obtain the key words of each abstract. We have also begun pre-processing our datasets to allow for keyword extraction. We still need to write the code based off of our pseudo code, extract the keywords from our data sets, evaluate the results from the keyword extraction, and discuss the results in our report. The schedule for our remaining work goes as follows:

\begin{table}[h]
\begin{center}
\begin{tabular}{|p{3.3cm}|p{1.4cm}|p{1.7cm}|}
\hline \bf Task & \bf Target Date & \bf Person  Responsible \\  
\hline
 Create code to extract keywords & Nov 13 & All \\ 
\hline 
 Extract keywords from datasets & Nov 17 & All \\
\hline
 Evaluate results & Nov 24 & All \\
\hline
Complete report writing & Dec 6 & All  \\
\hline   
\end{tabular}
\end{center}
\end{table}

As for the distribution of work currently completed, Shouyang has created the code for our pre-processing of our data sets. Sharif summarized the papers in the Related Works section. Sharon formatted this paper. We all had a hand in writing this draft of the report. 

% include your own bib file like this:
%\bibliographystyle{acl}
%\bibliography{acl2017}
\bibliography{report}
\bibliographystyle{acl_natbib}




\end{document}
